\documentclass[a4paper,10.5pt]{report}
\usepackage[T1]{fontenc}
\usepackage[utf8]{inputenc}
\usepackage[french]{babel}
\usepackage{empheq}
\usepackage{mathtools, bm}
\usepackage{amssymb, bm}
\usepackage{graphicx}
\usepackage{caption}
\usepackage{subcaption}
\usepackage{hyperref}

\usepackage{multicol}
\setlength{\columnsep}{2cm}

\title{\textbf{\Huge  Université Paris Sud}\\ Base de donnée, partie II}
\author{Coquisart Jérôme, Alcántara Hernández Úrsula}
\date{Année 2020}


\begin{document}
    \maketitle
    \newpage
    %\tableofcontents
    \newpage

    \section*{Introduction}
    \section{Étape 3: Fusion}

	ATERLIER (\underline{Num\_At}, Id\_Act NOT NULL, Jour NOT NULL, Créneau NOT NULL, Étage NOT NULL, Num\_S NOT NULL, Id\_Bat\_Local NOT NULL, N\_SS\_Anime NOT NULL, Niveau, Min, Max) \\

	ACTIVITÉ (\underline{Id\_Act}, Num\_dA NOT NULL, Nom\_Act, Pratique) \\

	HABITANT (\underline{N\_SS, }, Id\_Bat\_Im NOT NULL, Num\_AP NOT NULL, Prén\_Hab, Nom\_Hab, Genre, Date\_Nat)  \\

	SALLE (\underline{Étage}, \underline{Num\_S}, \underline{Id\_Bat\_Local}) \\

	APPART (\underline{Num\_AP}, \underline{Num\_S}, \underline{Id\_Bat\_Im}) \\

	LOCAL (\underline{Id\_Bat\_Local}, Nom\_Bat, Num\_Rue, Rue, CP, Type) \\

	IMMEUBLE (\underline{Id\_Bat\_Im}, Nom\_Bat, Nom\_Rue, Rue, CP, Année) \\

	HORAIRE (\underline{Jour}, \underline{Créneau}) \\

	DISC.ART\. (\underline{Num\_DA}, Nom\_DA) \\

	R\_PARTICIPE (\underline{N\_SS}, Num\_AT NOT NULL) \\

	R\_PARRAINE (\underline{N\_SS\_PARRAIN}, N\_SS\_Filleul NOT NULL) \\

	R\_LIEU\_POSSIBLE (\underline{Capacité}, Num\_DA NOT NULL, Étage NOT NULL, Num\_S NOT NULL, Id\_Bat\_Local NOT NULL) \\

	ATELIER[Id\_Act] $\subseteq$ ACTIVITÉ[Id\_Act] \\

	ATELIER[Jour] $\subseteq$ HORAIRE[Jour] \\

	ATELIER[Creneau] $\subseteq$ HORAIRE[Creneau] \\

	ATELIER[Etage] $\subseteq$ SALLE[Etage] \\

	ATELIER[NumS] $\subseteq$ SALLE[NumS] \\

	ATELIER[Id\_Bat\_Local] $\subseteq$ SALLE[Id\_Bat\_Local] \\

	ATELIER[NSS\_Anime] $\subseteq$ HABITANT[NSS] \\

	ACTIVITÉ[Num\_DA] $\subseteq$ DISC.ART.[Num\_DA] \\

	HABITANT[Id\_Bat\_Im] $\subseteq$ BATIMENT[Id\_Bat\_Im] \\

	HABITANT[Num\_AP] $\subseteq$ APPART[Num\_AP] \\

	SALLE[Id\_Bât\_Local] $\subseteq$ LOCAL[Id\_Bât\_Local] \\

	APPART[Id\_Bât\_Im] $\subseteq$ IMMEUBLE[Id\_Bât\_Im] \\

	R\_PARTICIPE[N\_SS]  $\subseteq$ HABITANT[N\_SS] \\

	R\_PARTICIPE[Num\_AT] $\subseteq$ ATELIER[Num\_AT] \\

	R\_PARRAINE[N\_SS\_Filleul] $\subseteq$ HABITANT[N\_SS] \\

	R\_PARRAINE[N\_SS\_Parrain] $\subseteq$ HABITANT[N\_SS] \\

	R\_LIEU\_POSSIBLE[Num\_DA] $\subseteq$ DISC.ART.[Num\_DA] \\

	R\_LIEU\_POSSIBLE[Etage] $\subseteq$ SALLE[Etage] \\

	R\_LIEU\_POSSIBLE[Num\_S] $\subseteq$ SALLE[Num\_S] \\

	R\_LIEU\_POSSIBLE[Id\_Bat\_Local] $\subseteq$ LOCAL[Id\_Bat\_Local] \\

	\textbf{Justifications: }
	\begin{itemize}
		\item	On fusionne les entités ATELIER avec les associations R\_THÈME, R\_ANIME et R\_A\_LIEU car la cardinalité minimum de l'entité ATELIER pour ces associations est 1. On ajoute les attributs nécessaires (les clés de ACTIVITÉ, HABITANT, HORAIRES et SALLE).
		\item	On fusionne l'entité ACTIVITÉ avec l'associations R\_DISCIPLINE car la cardinalité minimum de l'entité ATELIER pour cette association est 1. On ajoute les attributs nécessaires (les clés de DISC\. ART\.).
		\item	On fusionne l'entité HABITANT avec l'associations R\_HABITE car la cardinalité minimum de l'entité HABITANT pour cette association est 1. On ajoute les attributs nécessaires (les clés de APPART).
		\item On fusionne les ISA, c'est-à-dire LOCAL et BÂTIMENT deviennent LOCAL avec les attributs de bâtiment et de local. IMMEUBLE et BÂTIMENT avec les attributs de bâtiment et de immeuble.
		\item	On fusionne l'entité SALLE avec l'associations R\_DANS\_LOCAL car la cardinalité minimum de l'entité SALLE pour cette association est 1. On ajoute les attributs nécessaires (les clés de LOCAL et donc de BÂTIMENT car c'est une entité faible). 
		\item	On fusionne l'entité APPART avec l'associations R\_DANS\_2 car la cardinalité minimum de l'entité APPART pour cette association est 1. On ajoute les attributs nécessaires (les clés de IMMEUBLE et donc de BÂTIMENT car c'est une entité faible).
		\item Les associations R\_PARTICIPE, R\_LIEU\_POSSIBLE et R\_PARRAINE ne changent pas car les cardinalités minimales sont 0, pour les entitées associées.

	\end{itemize}
	\newpage
    \section{EN SQL}

		\begin{tabbing}
			CR\=EATE TABLE Atelier (\\
			\> Num$\_$AT INTEGER PRIMARY KEY,\\
			\> Id$\_$Act INTEGER NOT NULL REFERENCES Activité(Id\_Act),\\
			\> Jour DATE NOT NULL,\\
			\> Créneau DATETIME NOT NULL,\\
			\> Étage INTEGER NOT NULL,\\
			\> Num$\_$S INTEGER NOT NULL,\\
			\> Id$\_$Bât$\_$Local INTEGER NOT NULL,\\
			\> N$\_$SS$\_$Anime  INTEGER NOT NULL,\\
			\> Niveau INTEGER,\\
			\> Min INTEGER,\\
			\> Max INTEGER\\);
		\end{tabbing}

		\begin{tabbing}
			CR\=EATE TABLE Activité (\\
			\> Id\_Act INTEGER PRIMARY KEY,\\
			\> Num\_DA INTEGER NO NULL,\\
			\> Nom\_Act CHAR(30),\\
			\> Pratique CHAR(10),\\);
		\end{tabbing}

		\begin{tabbing}
			CR\=EATE TABLE Habitant (\\
			\> N\_SS INTEGER PRIMARY KEY,\\
			\> Id\_Bat\_Im INTEGER NOT NULL,\\
			\> Num\_AP INTEGER NOT NULL,\\
			\> Prén\_Hab CHAR(30),\\
			\> Nom\_Hab CHAR(30),\\
			\> Genre CHAR(1),\\
			\> Date\_Nat DATE,\\);
		\end{tabbing}

		\begin{tabbing}
			CR\=EATE TABLE Salle (\\
			\> Étage INTEGER PRIMARY KEY,\\
			\> Num\_S INTEGER NOT NULL,\\
			\> Id\_Bat\_Local INTEGER NOT NULL,\\);
		\end{tabbing}

		\begin{tabbing}
			CR\=EATE TABLE Appart (\\
			\> Num\_AP INTEGER	 PRIMARY KEY,\\
			\> Pièces INTEGER NOT NULL,\\
			\> Id\_Bat\_Im INTEGER NOT NULL,\\);
		\end{tabbing}
                                                   
		\begin{tabbing}
			CR\=EATE TABLE Local (\\
			\> Id\_Bat\_Local INTEGER PRIMARY KEY,\\
			\> Nom\_Bat CHAR(30),\\
			\> Nom\_Rue CHAR(30),\\
			\> Rue CHAR(40),\\
			\> CP INTEGER,\\
			\> Type CHAR(40),\\);
		\end{tabbing}
		\begin{tabbing}
			CR\=EATE TABLE Immeuble (\\
			\> Id\_Bat\_Im INTEGER PRIMARY KEY,\\
			\> Nom\_Bat CHAR(30),\\
			\> Nom\_Rue CHAR(30),\\
			\> Rue CHAR(40),\\
			\> CP INTEGER,\\
			\> Année YEAR\\);
		\end{tabbing}

		\begin{tabbing}
			CR\=EATE TABLE Horaire (\\
			\> Jour DATE PRIMARY KEY,\\
			\> Créneau DATE TIME NOT NULL,\\);
		\end{tabbing}
		\begin{tabbing}
			CR\=EATE TABLE Discipline\_Artistique (\\
			\> Num\_DA INTEGER PRIMARY KEY,\\
			\> Nom\_DA, CHAR(30)\\);
		\end{tabbing}
		\begin{tabbing}
			CR\=EATE TABLE R\_Participe (\\
			\> N\_SS INTEGER PRIMARY KEY,\\
			\> Num\_AT INTEGER NOT NULL,\\);
		\end{tabbing}
		\begin{tabbing}
			CR\=EATE TABLE R\_Parraine (\\
			\> N\_SS\_Parrain INTEGER PRIMARY KEY,\\
			\> N\_SS\_Filleul INTEGER NOT NULL,\\);
		\end{tabbing}
		\begin{tabbing}
			CR\=EATE TABLE R\_Lieu\_Possible (\\
			\> Capactité INTEGER PRIMARY KEY,\\
			\> Num\_DA INTEGER NOT NULL,\\
			\> Étage INTEGER NOT NULL,\\
			\> Num\_S INTEGER NOT NULL,\\
			\> Id\_Bat\_Local INTEGER NOT NULL,\\);
		\end{tabbing}


\end{document}
